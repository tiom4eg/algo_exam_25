\section{Быстрое преобразование Фурье}


\subsection{Прямое преобразование}
\label{forward_fft}

Пусть $P(x) \in \mathbb{C}[x]$ -- некоторой многочлен с комплексными коэффициентами.
Хотим вычислить значения $P(x)$ в корнях $x^n - 1$ (то есть корнях из единицы), для $n = 2^k$.

Будем делать это рекурсивно. Пусть требуется вычислить значения многочлена $P(x)$ в корнях $x^{n} - z$, где $z \in \mathbb{C} \setminus \{0\}$ и $\deg P(x) < n$
\begin{itemize}
    \item Если $n = 1$, то многочлен $P(x)$ -- константа, а единственный корень $x^n - z = x - z$ это $z$. Значит $P(z) = [x^0]P(x)$

    \item          Если $n \neq 1$, то $n = 2k$ для некоторого натурального $k$. Разложим $x^n - z$ как
          \begin{gather}
              x^n - z = x^{2k} - z = (x^k - \sqrt{z})(x^k + \sqrt{z})
          \end{gather}
          Очевидно, что множество корней $x^{2k} - z$ это в точности объединение множеств корней $x^k - \sqrt{z}$ и $x^k + \sqrt{z}$.
          Вычислим $P_1(x) = P(x) \bmod (x^k - \sqrt{z})$ и $P_2(x) = P(x) \bmod (x^k + \sqrt{z})$ (это делается за $O(n)$).
          Затем рекурсивно вычислим значения $P_1(x)$ в корнях $x^k - \sqrt{z}$ и значения $P_2(x)$ в корнях $x^k + \sqrt{z}$
\end{itemize}

При реализации стоит заранее вычислить значение константы $\sqrt{z}$ для каждого рекурсивного вызова.
Например можно предподсчитать последовательность
$$
    w_n = \begin{cases}
        1                                           & \text{если $n = 0$}                              \\
        \cos \frac{\pi}{2n} + i \sin \frac{\pi}{2n} & \text{если $n = 2^k$}                            \\
        w_{n - 2^k} \cdot w_{2^k}                   & \text{иначе, где $k = \lfloor \log_2 n \rfloor$} \\
    \end{cases}
$$
и в $i$-ом рекурсивном вызову на уровне $d$ использовать $\sqrt{z} = w_i$ (в нумерации с нуля)


\subsection{Обратное преобразование}
\label{inverse_fft}

Для обращения преобразования достаточно обратить вышеописанный алгоритм.

Рассмотрим рекурсивный вызов вышеописанного алгоритма
\begin{itemize}
    \item Если $n = 1$, то обращать нечего, так как никаких вычислений не выполнялось
    \item Если $n \neq 1$, то $n = 2k$ для некоторого натурального $k$. Заметим, что значение $P(x) \bmod (x^{2k} - z)$
          однозначно определяется значениями $P_1(x) = P(x) \bmod (x^k - \sqrt{z})$ и $P_2(x) = P(x) \bmod (x^k + \sqrt{z})$, так как многочлены $x^k - \sqrt{z}$ и $x^k + \sqrt{z}$ взаимно просты.

          Вспомним формулу для явного нахождения решения системы сравнений из китайской теоремы об остатках:
          \begin{gather}
              \begin{cases}
                  X \equiv A \quad (\text{mod } C) \\
                  X \equiv B \quad (\text{mod } D) \\
              \end{cases} \\
              %   \implies
              X \equiv_{CD}
              A \cdot \text{inv}(D, C) \cdot D +
              B \cdot \text{inv}(C, D) \cdot C
          \end{gather}
          где $\text{inv}(x, y)$ -- обратное к $x$ по модулю $y$

          Подставим $X = P(x),\ A = P_1(x),\ B = P_2(x),\ C = x^k - \sqrt{z},\ D = x^k + \sqrt{z}$

          Тогда $\text{inv}(D, C) = \text{inv}(D \bmod C, C) = \text{inv}(2\sqrt{z}, C) = \frac{1}{2\sqrt{z}}$ и $\text{inv}(C, D) = -\frac{1}{2\sqrt{z}}$

          Значит
          \begin{gather}
              P(x) =
              P_1(x) \cdot \frac{1}{2\sqrt{z}} \left(x^k + \sqrt{z} \right) -
              P_2(x) \cdot \frac{1}{2\sqrt{z}} \left(x^k - \sqrt{z} \right) =
              \\
              \frac{1}{2} \left(
              P_1(x) \cdot  \left(1 + x^k \cdot \frac{1}{\sqrt{z}} \right) +
              P_2(x) \cdot  \left(1 - x^k \cdot \frac{1}{\sqrt{z}} \right)
              \right)
          \end{gather}

          что позволяет вычислить $P(x) \bmod x^n - z$ за $O(n)$
\end{itemize}


\subsection{Альтернативная интерпретация}


Вычисление $P_1(x)$ и $P_2(x)$ в \ref{forward_fft} это взятие $P(x)$ по модулю $x^k - \sqrt{z}$ и $x^k + \sqrt{z}$ соответственно.
Пусть $w = \sqrt{z}$.
Представим $P(x)$ в виде $P(x) = A(x) + x^k B(x)$ (то есть $A(x), B(x)$ -- младшая и старшая половина коэффициентов соответственно).
Тогда выполнено

\begin{@empty}
    \renewcommand*{\arraystretch}{1.7}

    \begin{equation}
        \begin{bmatrix}
            P_1(x) \\
            P_2(x) \\
        \end{bmatrix}
        =
        \begin{bmatrix}
            1 & w  \\
            1 & -w \\
        \end{bmatrix}
        \times
        \begin{bmatrix}
            A(x) \\
            B(x) \\
        \end{bmatrix}
    \end{equation}
\end{@empty}

Значит обратное преобразование может быть вычислено по формуле


\begin{@empty}
    \renewcommand*{\arraystretch}{1.7}

    \begin{equation}
        \begin{bmatrix}
            A(x) \\
            B(x) \\
        \end{bmatrix}
        =
        \begin{bmatrix}
            1 & w  \\
            1 & -w \\
        \end{bmatrix}^{-1}
        \times
        \begin{bmatrix}
            P_1(x) \\
            P_2(x) \\
        \end{bmatrix}
        =
        \frac{1}{2}
        \begin{bmatrix}
            1           & 1            \\
            \frac{1}{w} & -\frac{1}{w} \\
        \end{bmatrix}
        \times
        \begin{bmatrix}
            P_1(x) \\
            P_2(x) \\
        \end{bmatrix}
    \end{equation}
\end{@empty}

Прямое и обратное преобразование можно вычислять без использования дополнительной памяти,
просто сразу записывая на место $A(x)$ и $B(x)$ значения $P_1(x), P_2(x)$ и наоборот.


\subsection{Одновременное вычисление для двух многочленов из $\mathbb{R}[x]$}

Даны два многочлена $A(x), B(x) \in \mathbb{R}[x]$, хотим вычислить их значения в корнях $(x^n - 1)$ за одно преобразование Фурье размера $n$ ($n$ -- степень двойки).
Заметим, что для $P(x) \in \mathbb{R}[x]$ выполнено $P(\overline{z}) = \overline{P(z)}$.
Пусть $P(x) = A(x) + i \cdot B(x)$.
Пусть $w$ -- какой-то из корней $(x^n - 1)$.

% Пусть
% \begin{gather}
%     \begin{split}
%         A(w) = a_1 + i \cdot a_2  \qquad B(w) = b_1 + i \cdot b_2
%         \\
%         A(\overline{w}) = a_1 - i \cdot a_2  \qquad B(\overline{w}) = b_1 - i \cdot b_2
%     \end{split}
% \end{gather}
% Тогда



\begin{gather}
    P(w) = A(w) + i \cdot B(w) \qquad P(\overline{w}) = A(\overline{w}) + i \cdot B(\overline{w})
\end{gather}

\begin{gather}
    \begin{split}
        P(w) + \overline{P(\overline{w})} =
        \\
        A(w) + i\cdot B(w) + \overline{A(\overline{w}) + i \cdot B(\overline{w})} =
        \\
        A(w) + i\cdot B(w) + \overline{A(\overline{w})} + \overline{i \cdot B(\overline{w})} =
        \\
        A(w) + i\cdot B(w) + A(w) + \overline{i} \cdot \overline{B(\overline{w})} =
        \\
        A(w) + i\cdot B(w) + A(w) - i \cdot B(w) =
        \\
        2A(w)
    \end{split}
    \qquad \qquad
    \begin{split}
        P(w) - \overline{P(\overline{w})} =
        \\
        A(w) + i\cdot B(w) - \overline{A(\overline{w}) - i \cdot B(\overline{w})} =
        \\
        A(w) + i\cdot B(w) - \overline{A(\overline{w})} - \overline{i \cdot B(\overline{w})} =
        \\
        A(w) + i\cdot B(w) - A(w) - \overline{i} \cdot \overline{B(\overline{w})} =
        \\
        A(w) + i\cdot B(w) - A(w) + i \cdot B(w) =
        \\
        2i \cdot B(w)
    \end{split}
\end{gather}


Что позволяет зная $P(w), P(\overline{w})$ вычислить $A(w), B(w)$ за $O(1)$.
Все корни, кроме $\pm 1$, разбиваются на пары сопряжённых.


\subsection{Свёртка}

Хотим вычислить произведение многочленов $A(x), B(x) \in \mathbb{C}[x]$.
Пусть $k = \big\lceil \log_2 \left( \deg A(x) + \deg B(x) + 1 \right) \big\rceil$ и $n = 2^k$.
Пусть $C(x) = A(x)B(x)$ -- искомый многочлен.

Заметим, что так как $\deg C(x) = \deg A(x) + \deg B(x) < n$, то достаточно найти $C(x) \bmod (x^n - 1)$, так как $C(x) \bmod (x^n - 1) = C(x)$.
Значение $C(x) \bmod (x^n - 1)$ однозначно определяется значениями $C(x)$ в корнях $(x^n - 1)$.
Используя \ref{forward_fft} мы можем найти значения $A(x), B(x)$ в этим корнях, а затем и значения $C(x)$ в них, так как для любого $z \in \mathbb{C}$ выполнено $C(z) = A(z)B(z)$.
Затем, используя \ref{inverse_fft}, мы можем восстановить $C(x) \bmod (x^n - 1)$ по значениям в корнях



\subsection{Прикладывания}

Пусть даны две последовательности $a_0, a_1, a_2, ..., a_{n - 1}$  и $b_0, b_1, ..., b_{m - 1}$, причём $m \le n$.
Хотим для каждого $k$ от $0$ до $n - m$ найти сумму
\begin{gather}
    c_k = \sum_{i = 0}^{m - 1} a_{k + i} \cdot b_i
\end{gather}

То есть как бы *приложить* последовательность $b$ к последовательности $a$, и вычислить сумму поэлементных произведений.
Для этого просто вычислим произведение $A(x)B(x)$, где
\begin{gather}
    A(x) = \sum_{i = 0}^{n - 1} x^i \cdot a_i \qquad B(x) = \sum_{i = 0}^{m - 1} x^i \cdot b_{m - 1 - i}
\end{gather}
То есть свёртку последовательности $a$ с развёрнутой последовательностью $b$.

Заметим, что последовательные $n - m + 1$ коэффициентов, начиная с коэффициента при $x^{m - 1}$ и будет результатом.
Причём в свёртке можно использовать $k = \lceil \log_2 n \rceil$, а не $k = \lceil \log_2 \left(n + m - 1\right) \rceil$,
так как умножение по модулю $(x^n - 1)$ это *циклическая* свёртка, и *переполнение* не затронет результат, так как им будет затронуты не более чем $m - 2$ первых членов



\subsection{Поиск по шаблону}

Чтобы сделать поиск по шаблону, для каждого прикладывания вычислим
\begin{gather}
    \sum_{x, y} \left(x - y\right)^2
\end{gather}
где сумма берётся по всем парам прикладываемых к друг другу символов.
Сумма равна $0$ тогда и только тогда, когда все символы в прикладываемых парах совпадают.
Для вычисления суммы раскроем скобки и каждое слагаемое посчитаем отдельно (для слагаемых с $\deg_x > 0$ и $\deg_y > 0$ понадобится свёртка)

Если разрешается иметь отличие на не более чем 1, то аналогичным образом вычислим
\begin{gather}
    \sum_{x, y} \left(x - y - 1\right)^2 \left(x - y \right)^2 \left(x - y + 1\right)^2
\end{gather}

Такая сумма равна $0$, тогда и только тогда, когда все пары символов отличаются не более, чем на 1


