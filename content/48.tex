\section{Быстрое преобразование Фурье}

Пусть $P(x) \in \mathbb{C}[x]$ -- некоторой многочлен с комплексными коэффициентами.
Хотим вычислить значения $P(x)$ в корнях $x^n - 1$ (то есть корнях из единицы), для $n = 2^k$.

Будем делать это рекурсивно. Пусть требуется вычислить значения многочлена $P(x)$ в корнях $x^{n} - z$, где $z \in \mathbb{C} \setminus \{0\}$ и $\deg P(x) < n$
\begin{itemize}
    \item Если $n = 1$, то многочлен $P(x)$ -- константа, а единственный корень $x^n - z = x - z$ это $z$. Значит $P(z) = [x^0]P(x)$
    \item Если $n \neq 1$, то $n = 2k$ для некоторого натурального $k$. Разложим $x^n - z$ как $x^n - z = x^{2k} - z = (x^k - \sqrt{z})(x^k + \sqrt{z})$.
          Очевидно, что множество корней $x^{2k} - z$ это в точности объединений множеств корней $x^k - \sqrt{z}$ и $x^k + \sqrt{z}$.
          Вычислим $P_1(x) = P(x) \bmod (x^k - \sqrt{z})$ и $P_2(x) = P(x) \bmod (x^k + \sqrt{z})$ (это делается за $O(n)$).
          Затем рекурсивно вычислим значения $P_1(x)$ в корнях $x^k - \sqrt{z}$ и значения $P_2(x)$ в корнях $x^k + \sqrt{z}$.
\end{itemize}

При реализации стоит заранее вычислить значение константы $\sqrt{z}$ для каждого рекурсивного вызова.
Например можно предподсчитать последовательность
$$
    w_n = \begin{cases}
        1                                         & \text{если $n = 1$}                              \\
        \cos \frac{\pi}{n} + i \sin \frac{\pi}{n} & \text{если $n = 2^k$}                            \\
        w_{n - 2^k} \cdot w_{2^k}                 & \text{иначе, где $k = \lfloor \log_2 n \rfloor$} \\
    \end{cases}
$$
и в $i$-ом рекурсивном вызову на уровне $d$ использовать $\sqrt{z} = w_i$ (нумерация идёт с единицы)