\section{Тернарный поиск}

Пусть $f(x)$ имеет глобальный минимум в точке $x^*$,
причём строго убывает на $(-\infty, x^*]$ и строго возрастает на $[x^*, +\infty)$.
% Тогда верны следующие утверждения:
% \begin{itemize}
%     \item Пусть $m_0 < m_1$ и $f(m_0) \le f(m_1)$, тогда $x^* \notin (m_1, +\infty)$
%     \item Пусть $m_0 < m_1$ и $f(m_0) \ge f(m_1)$, тогда $x^* \notin (-\infty, m_0)$
% \end{itemize}

Пусть известно, что $x^* \in [l, r]$ (исходно положим $(l, r) =  (-C, C)$).
Пусть $m_0 = (2l + r) / 3$ и $m_1 = (l + 2r) / 3$ ($m_0, m_1$ делят отрезок $[l, r]$ в отношении $1:1:1$).
Вычислим значения $f(m_0), f(m_1)$
\begin{itemize}
    \item Если $f(m_0) \le f(m_1)$, то $x^* \in [l, m_1]$. Заменим $(l, r)$ на $(l, m_1)$
    \item Если $f(m_0) \ge f(m_1)$, то $x^* \in [m_0, r]$. Заменим $(l, r)$ на $(m_0, r)$
\end{itemize}
Повторим, пока не будет выполнено $r < l + \varepsilon$, где $\varepsilon$ -- требуемая точность

За итерацию длина отрезка $[l, r]$ уменьшается ровно в $1.5$ раза,
значит всего потребует не более чем $2 \log_{1.5} \frac{C}{\varepsilon} + O(1) = O(\log \frac{C}{\varepsilon})$ вычислений функции $f(x)$.


\subsection{Трюк с золотым сечением}

Пусть $\varphi = \frac{1 + \sqrt{5}}{2} \approx 1.618$ -- корень уравнения $x^2 - x - 1 = 0$.
Будем брать
$m_0 = \frac{1}{2\varphi + 1}\left((\varphi + 1) \cdot l + \varphi \cdot r \right)$
и
$m_1 = \frac{1}{2\varphi + 1}\left(\varphi \cdot l + (\varphi + 1) \cdot r \right)$
($m_0, m_1$ делят отрезок $[l, r]$ в отношении $ \varphi : 1 : \varphi$).

Тогда, так как $\frac{1 + 2\varphi}{1 + \varphi} = \varphi$, то на каждой итерации вышеописанного алгоритма, кроме первой,
одно из значений $f(m_0), f(m_1)$ будет уже вычислено.
Причём длина отрезка $[l, r]$ будет уменьшаться в $\frac{1 + 2\varphi}{1 + \varphi} = \varphi$ раз.
Поэтому суммарно будет произведено $\log_{\varphi} \frac{C}{\varepsilon} + O(1)$ вычислений функции $f(x)$.
Что примерно в $\frac{2\ln(1.618)}{\ln(1.5)} \approx 2.37$ раз меньше $2\log_{1.5} \frac{C}{\varepsilon}$