\section{Median of Medians/Quickselect}

По сути, мы хотим находить такую разделяющую точку, что она будет всегда работать достаточно хорошо. Для этого, разобьём все элементы на блоки по 5 элементов, в них отсортируем элементы (по сути, за $O(1)$), далее найдём медиану среди всех медианных элементов в блоках. Заметим, что будет выполнено следующее: в тех блоках, в которых медиана будет меньше медианы медиан, первые три элемента также гарантированно будут меньше, то есть про них мы можем сказать, что они точно окажутся слева от разделяющего элемента. Аналогично, в блоках, в которых медиана больше медианы медиан, последние три элемента гарантированно окажутся справа. Поскольку в обоих случаях блоков будет $\frac{3n}{10}$, получаем, что размер каждой части разбиения будет находиться между $\frac{3n}{10}$ и $\frac{7n}{10}$. Предположим, что каждый раз мы будем попадать в худший из случаев и идти в блок размера $\frac{7n}{10}$. Тогда, $T(n) = T(\frac{n}{5}) + T(\frac{7n}{10}) + O(n) = O(n)$ (по мастер-теореме)